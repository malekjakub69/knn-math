\section{Úvod}
Tento dokument popisuje problematiku a řešení rozpoznání matemtických rovnic. Jedná se o velmi aktuální téma, které se v posledních letech stalo předmětem mnoha výzkumů a aplikací. Rozpoznávání matematických vzorců je důležité pro automatizaci různých úloh, jako je převod tištěných nebo ručně psaných vzorců do digitálního formátu, což usnadňuje jejich zpracování a analýzu.

% Tento dokument popisuje strukturu a implementaci projektu KNN-Math, který se zaměřuje na rozpoznávání matematických vzorců pomocí algoritmu k-nejbližších sousedů (KNN). Systém je navržen pro rozpoznávání jak tištěných, tak ručně psaných matematických výrazů a jejich převod do formátu LaTeX.

\section{Popis úlohy}
Úkolem projektu je vytvoření a natrénování modelu pomocí strojového učení, který bude schopný rozpoznat obrázky matematických výrazů na vstupu a převést je do strojového zápisu. Formát vstupních obrázků jsme si zvolili jako obrázky ručně psaných rovnic, které budou následně převedeny do formátu LaTeX -- \textbf{offline recognition}.


\section{Přehled existujících řešení}
V celé historii problematiky rozpoznávání ručne psaných matematických rovnic se objevily tři odlišné přístupy~\ref{img:method_history}. Na počátku se jednalo o jednodušší více přímočaré metody označované jako sekvenční řešení~\cite{ukr_survey}.

% ===== Method histoty img =====
\begin{figure}[H]
    \centering
    \includegraphics[width=0.6\textwidth]{img/method_history.png}
    \caption{Časový vývoj různých metod pro rozpoznávání matematických vzorců}
    \label{img:method_history}
\end{figure}

\textbf{Sekvenční řešení} použí princip dekompozice a tuto metodu lze rozložit do několika částí (obrázek~\ref{img:seq_sol}). Modely pracující na tomto principu nejprve (1) rozpoznají jednotlivé symboly a následeně provedou (2) strukturní analýzu, kde jednotlivé symboly poskládají do celého výrazu. Rozpoznávání symbolů lze pak dále rozložit na segmentaci jednotlivých symbolů a jejich klasifikaci. Tato metoda o několika krocích má ale nevýhodu v akumulaci chyb během výpočtu jeddnotlivých kroků.

% ===== Seq. Sol. img =====
\begin{figure}[H]
    \centering
    \includegraphics[width=0.6\textwidth]{img/sequential_sol.png}
    \caption{Diagram postupu sekvenčního řešení}
    \label{img:seq_sol}
\end{figure}

\textbf{Integrovaná řešení}: generují množinu symbolických hypotéz a následně pomocí strukturní analýzy vyberou tu nejlepší na základě znalosti odpovídající gramatiky a sémantiky. Tento přístup je složitější a vyžaduje více výpočetních prostředků, ale může poskytnout lepší výsledky v případě složitějších vzorců.

Zatímco předchozí dvě metody řešení problematiky rozpoznávání matamatických rovnic byly v posledních dekádách studovány důkladně, \textbf{end-to-end řešení} na principu \textbf{encoder--decoder} se začaly objevovat relativně nedávno. Tento přístup se snaží vyřešit problém rozpoznávání vzorců jako celek, bez rozdělení na jednotlivé části. Používá hluboké učení a neuronové sítě bez jakékoliv znalosti konkrétní domény pro tvorbu modelu -- \textbf{data-driven} přístup. Enkodér převede vstup na latentní reprezentaci a dekodér extrahuje skrytou reprezentaci za pomoci attention mechanismu a generuje výstupní sekvenci. 

Jeden z prvních návhů konktrétní architektury se skládal z VGG sítě, která sloužila jako enkodér \cite{ZHANG_1}. Model dále využíval GRU jako dekodér. Takto navržený systém disponuje přesností okolo 45~\% na datasetech CROHME (Competiton on Recognition of Handwritten Mathematical Expressions (HME)), které jsou standardem evaluace modelů pro rozpoznávání matematických vzorců. Použitá evaluační metrika je expression recognition rate/accuracy, která vyhodnocuje správnost modelu jak z pohledu jednotlivých symbolů, tak i celkové struktury. Jako správný se výstup považuje v případě, že celkový počet chyb v jednom výstupu není více než 3. 

Vylepšení původního návrhu architektury dále zahrnovaly nahrazení VGG sítě a použité DenseNet (densely connected conv. network) jako enkodéru \cite{ZHANG_2}. Ve článku \cite{LeAnhDuc} představují end-to-end řešení: kaskáda CNN - feature extractor, BLSTM - encoder a LSTM s attention mechanismem jako dekodér.

V posledním ročníku soutěže CROHME 2023 se objevilo několik návrhů. Vítězný tým představil model skládající se z vrstev CNN a BLSTM pro enkodér a dekodér v podobě attention mechanismu (expression recognition rate 86.95~\%).

Další účastnící se řešení používá DenseNet jako enkodér a 2 různé řešen dekodéru: bidirectional tree decoder a klasiký dekodér. S tímto přístupem vznikají 2 modely, jejichž spojením vzniká hybridní model (expression recognition rate 86.95~\%). 

Poslední model představený v soutěži, který uvedeme v této kapitole je složen ze 2 částí: obrazový model -- CRNN architektura prénonvaná na CTC loss funkci, a jazykový model -- LSTM architektura trénovaná pro predikci dalšího LaTeX tokenu (expression recognition rate 75.55~\%).

Týmy mají pro tvorbu řešení k dispozici online i offline datasety, které důmyslně používají spolu s augmentací a generování offline dat z online datasetu pro zlepšení výsledků.

\cite{CROHME_2023}



\section{Popis vlastního řešení}

\subsection{Experimenty}

\subsection{Výsledky \~~Evaluace (Kvantitativní vs. kvalitativní)}	

\section{Závěr \~~Takeaway messages}


\newpage
\section{Závěr}
Tato dokumentace popisuje strukturu a implementaci projektu KNN-Math. Projekt je navržen modulárně, což umožňuje snadnou údržbu a rozšiřování. Implementovaný systém používá algoritmus k-nejbližších sousedů pro rozpoznávání matematických symbolů a převod do LaTeX formátu.





\section{Kuba CHATGPT}
Tak víc než toto nemám :/ 

Jistě, zde je shrnutí průběhu experimentů a výsledků:

---

První pokusy (malý batch, limit dat, základní augmentace),
,
Batch size: 16,
Epochs: 5,
Limit: 1000 vzorků,
Výsledek:
Train accuracy: až 1.0 (100\%),
Train loss: 0.4592,
Val accuracy: 0.0355 (3.55\%),
Val loss: 5.1749,
,
Závěr: Model se naučil trénovací data nazpaměť, ale na validaci selhával (overfitting).,

---

Zvětšení batch, více dat, zvýšený dropout, regularizace,
,
Batch size: 32–128,
Dropout: 0.7 → 0.5,
Weight decay: 0.01–0.02,
Label smoothing: 0.1–0.3,
Learning rate: 1e-4 → 1e-5 (příliš nízký nefungoval),
Výsledek:
Train accuracy: 0.3141 (31.41\%) po 4 epochách (batch 128),
Val accuracy: 0.0178 (1.78\%),
Train loss: 1.1588,
Val loss: 7.5405,
,
Závěr: Overfitting se snížil, ale validace stále velmi slabá.,

---

Oprava resize a paddingu, fixní výška 80, šířka 256,
,
Vstupní obrázky: [3, 80, 256],
Batch size: 128,
Epochs: 5 (test), plánováno 67–100 (celonoční běh),
Výsledek:
Epoch 1:
Train loss: 1.0333,
Train accuracy: 0.3446,
Val loss: 6.2991,
Val accuracy: 0.0000,
,
Epoch 2:
Train loss: 0.8438,
Train accuracy: 0.4296,
Val loss: 6.7513,
Val accuracy: 0.0178,
,
,
Závěr: Model se učí, trénovací přesnost roste, validace je stále velmi nízká, ale už není nulová.,

Technické úpravy
Opraven resize s poměrem stran,
Padding na fixní šířku,
Opraveno stackování batchů,
Opraven import a použití F.pad





\begin{comment}

% %%%%%%%%%%%%%%%%%%%%%%%%%%%%%%%%%%%%%%%%%%%%%%%%%%
% %%%%%%%%%%%%%%%%% Stará verze %%%%%%%%%%%%%%%%%%%%
% %%%%%%%%%%%%%%%%%%%%%%%%%%%%%%%%%%%%%%%%%%%%%%%%%%

\section{Struktura projektu}

Projekt je organizován do modulárního systému, který usnadňuje údržbu a rozšiřitelnost. Hlavní adresářová struktura je následující:


\section{Popis modulů}

\subsection{Modul main.py}

Tento modul slouží jako hlavní vstupní bod aplikace a poskytuje rozhraní příkazové řádky pro různé operace:
\begin{itemize}
    \item Trénování modelu pomocí trénovacích dat
    \item Testování modelu na testovacích datech
    \item Rozpoznávání matematických výrazů v jednotlivých obrázcích
\end{itemize}

\subsection{Modul preprocessing}

Tento modul se stará o přípravu vstupních obrázků pro rozpoznávání:

\begin{itemize}
    \item \textbf{segmentation.py} - Obsahuje funkce pro segmentaci matematických symbolů z celého obrázku, převod obrázků do stupňů šedi, prahování pro oddělení symbolů od pozadí a normalizaci velikosti obrázků na standardní rozměr.
\end{itemize}

\subsection{Modul recognition}

Tento modul implementuje algoritmus rozpoznávání symbolů:

\begin{itemize}
    \item \textbf{knn\_classifier.py} - Implementuje KNN klasifikátor pro rozpoznávání jednotlivých symbolů. Zahrnuje funkce pro trénování modelu, predikci symbolů, ukládání a načítání natrénovaného modelu.
\end{itemize}

\subsection{Modul parsing}

Tento modul převádí rozpoznané symboly do LaTeX formátu:

\begin{itemize}
    \item \textbf{latex\_converter.py} - Obsahuje mapování mezi rozpoznanými symboly a jejich LaTeX reprezentací, a funkci pro sestavení výsledného LaTeX kódu z posloupnosti symbolů.
\end{itemize}

\subsection{Modul utils}

Tento modul poskytuje pomocné funkce:

\begin{itemize}
    \item \textbf{data\_loader.py} - Funkce pro načítání obrázků a popisků z datasetu.
    \item \textbf{evaluation.py} - Funkce pro vyhodnocení přesnosti a výkonnosti modelu.
\end{itemize}

\subsection{Modul config.py}

Konfigurační soubor obsahující globální parametry:

\begin{itemize}
    \item Cesty k adresářům (dataset, modely)
    \item Parametry KNN klasifikátoru (počet sousedů, váhy)
    \item Parametry pro předzpracování obrázků (velikost po normalizaci)
\end{itemize}

\section{Algoritmus rozpoznávání}

Algoritmus rozpoznávání matematických výrazů v aplikaci probíhá v následujících krocích:

\begin{enumerate}
    \item \textbf{Načtení vstupního obrázku} - Obrázek je načten a převeden do stupňů šedi.
    
    \item \textbf{Segmentace symbolů} - Obrázek je prahován pro oddělení symbolů od pozadí. Následně jsou nalezeny kontury jednotlivých symbolů a extrahovány jako samostatné obrázky.
    
    \item \textbf{Předzpracování symbolů} - Každý extrahovaný symbol je normalizován na standardní velikost a provedeny další úpravy pro zvýšení přesnosti rozpoznávání.
    
    \item \textbf{Klasifikace symbolů} - KNN klasifikátor predikuje identitu každého symbolu na základě natrénovaného modelu.
    
    \item \textbf{Konverze do LaTeXu} - Posloupnost rozpoznaných symbolů je převedena do LaTeX formátu, který zachovává matematickou strukturu výrazu.
\end{enumerate}

\section{KNN klasifikátor}

Pro rozpoznávání jednotlivých symbolů je použit algoritmus k-nejbližších sousedů (KNN). Princip tohoto algoritmu je následující:

\begin{itemize}
    \item Každý obrázek je reprezentován jako vektor příznaků (v našem případě jako vektor hodnot pixelů).
    \item Pro klasifikaci nového symbolu je vypočtena vzdálenost mezi jeho vektorem příznaků a vektory všech trénovacích vzorků.
    \item Je vybráno $k$ trénovacích vzorků s nejmenší vzdáleností.
    \item Nový symbol je klasifikován do třídy, která se nejčastěji vyskytuje mezi těmito $k$ nejbližšími sousedy.
\end{itemize}

V implementaci používáme weighted KNN, kde vzdálenější sousedé mají menší vliv na konečnou klasifikaci.

\section{Dataset a trénování}

Pro trénování a testování modelu je použit dataset matematických symbolů. Dataset obsahuje tisíce obrázků v adresáři \texttt{dataset} a soubor \texttt{train\_labels.txt}, který obsahuje mapování mezi názvy souborů a odpovídajícími symboly.

Proces trénování zahrnuje:
\begin{enumerate}
    \item Načtení obrázků a jejich štítků
    \item Předzpracování obrázků (normalizace velikosti, převod do stupňů šedi, atd.)
    \item Extrakce příznaků (v našem případě příznaky jsou přímo hodnoty pixelů)
    \item Trénování KNN klasifikátoru
    \item Uložení natrénovaného modelu pro pozdější použití
\end{enumerate}

\section{Použití aplikace}

Aplikaci lze použít několika způsoby:

\subsection{Trénování modelu}

\begin{lstlisting}[language=bash]
python -m app.main --train
\end{lstlisting}

\subsection{Testování modelu}

\begin{lstlisting}[language=bash]
python -m app.main --test
\end{lstlisting}

\subsection{Rozpoznávání matematického výrazu v obrázku}

\begin{lstlisting}[language=bash]
python -m app.main --recognize cesta/k/obrazku.png --output vystup.tex
\end{lstlisting}

\section{Možná rozšíření}

Navržená struktura projektu umožňuje snadné rozšíření o další funkce:

\begin{itemize}
    \item \textbf{Lepší algoritmy segmentace} - Implementace pokročilejších metod pro segmentaci složitějších matematických struktur (zlomky, mocniny, integrály).
    
    \item \textbf{Pokročilejší klasifikátory} - Nahrazení KNN klasifikátoru konvolučními neuronovými sítěmi nebo jinými algoritmy strojového učení.
    
    \item \textbf{Rozpoznávání struktury} - Implementace algoritmů pro rozpoznávání struktury matematických výrazů, nejen jednotlivých symbolů.
    
    \item \textbf{Webové rozhraní} - Vytvoření webového rozhraní pro snadné použití aplikace.
\end{itemize}
\end{comment}